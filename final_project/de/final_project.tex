\documentclass[12pt,a4paper]{article}
\usepackage[margin=2.5cm]{geometry}
\usepackage[utf8]{inputenc}
\usepackage[ngerman]{babel}
\usepackage{hyperref}
\usepackage{enumitem}
\usepackage{fancyhdr}

\pagestyle{fancy}
\fancyhf{}
\lhead{\textbf{Abschlussprojekt: Multimodales KI-System für Marktanalysen}}
\rfoot{\thepage}

\begin{document}

\begin{center}
    \LARGE\textbf{Abschlussprojekt: Multimodales KI-System für Marktanalysen}\\[0.5cm]
    \large Intensivkurs zu Generativer KI \\[0.3cm]
\end{center}

\section*{Projektübersicht}

Studierende entwickeln ein multimodales KI-System, das in der Lage ist, marktbezogene Fragen zu beantworten, Anlageeinsichten zu liefern, historische Marktdaten zu analysieren, Prognosen zu erstellen und Finanzdaten zu visualisieren. Das System besteht aus spezialisierten KI-Agenten, die koordiniert zusammenarbeiten und reale Finanzdaten nutzen, welche ausschließlich aus offiziellen \textbf{Investor-Relations (IR)}-Dokumenten von \textbf{Apple, Microsoft, Google, NVIDIA und Meta} aus den Jahren \textbf{2020 bis 2024} stammen.

\section*{Systemarchitektur und Rollen der Agenten}

Das System basiert auf einem Multi-Agenten-Rahmenwerk, bestehend aus den folgenden klar definierten Agenten:

\subsection*{1. Multimodaler Agentischer RAG-Spezialist}

\textbf{Hauptaufgaben:}
\begin{itemize}[noitemsep]
    \item Verarbeitung multimodaler Abfragen (Text, Tabellen, Bilder, Diagramme, PDFs).
    \item Abrufen relevanter Finanzdaten aus den IR-Dokumenten.
    \item Erstellung präziser Antworten mit expliziten Quellenangaben.
\end{itemize}

\textbf{Wichtige Technologien:}
\begin{itemize}[noitemsep]
    \item Embedding-Modelle: CLIP, SentenceTransformers
    \item Vektordatenbanken: Chroma
    \item Generative Modelle: Gemini (optional mit QLoRA Fine-tuning)
\end{itemize}

\textbf{Beispielabfrage:}

\textit{„Fassen Sie die aktuelle Finanzleistung von NVIDIA basierend auf dieser Investorenpräsentation zusammen.“}

\textbf{Beispielantwort:}

\textit{„NVIDIAs Umsatz stieg im Q4 FY24 um 18\,\%, hauptsächlich durch starke GPU-Verkäufe (Quelle: NVIDIA Q4 FY24 Earnings Slides, Seite 5).“}

\subsection*{2. Data-Science- und Analyse-Agent}

\textbf{Hauptaufgaben:}
\begin{itemize}[noitemsep]
    \item Durchführung detaillierter Trendanalysen und Marktprognosen.
    \item Erstellung visueller Darstellungen und Vorhersagemodelle.
    \item Generierung erklärender Texte zu Analyseergebnissen.
\end{itemize}

\textbf{Wichtige Technologien:}
\begin{itemize}[noitemsep]
    \item Datenanalyse: Pandas, scikit-learn
    \item Prognosemodelle: Prophet, ARIMA
    \item Visualisierung: Matplotlib, Plotly
    \item Generative Modelle: Gemini (zur Generierung erklärender Texte)
\end{itemize}

\textbf{Beispielabfrage:}

\textit{„Analysieren Sie die Kursentwicklung von Microsoft im letzten Jahr und prognostizieren Sie die Entwicklung im nächsten Quartal.“}

\subsection*{3. Websuche- und Echtzeitmarkt-Agent}

\textbf{Hauptaufgaben:}
\begin{itemize}[noitemsep]
    \item Abrufen aktueller Marktnachrichten, Stimmungen und Ereignisse.
    \item Zusammenfassung aktueller Finanzinformationen mit Quellenangaben.
\end{itemize}

\textbf{Wichtige Technologien:}
\begin{itemize}[noitemsep]
    \item Web-APIs: SerpAPI, Tavily, NewsAPI
    \item Web Scraping: BeautifulSoup, newspaper3k
    \item Summarization APIs: Gemini, Hugging Face
\end{itemize}

\textbf{Beispielabfrage:}

\textit{„Welche aktuellen Ereignisse beeinflussen heute den Aktienkurs von Google?“}

\subsection*{4. Koordinations-Agent}

\textbf{Hauptaufgaben:}
\begin{itemize}[noitemsep]
    \item Zerlegung multimodaler Abfragen in Teilaufgaben.
    \item Koordination der Zusammenarbeit zwischen den einzelnen Agenten.
    \item Aggregation der Ergebnisse zu einer kohärenten Analyse mit Quellenangaben.
\end{itemize}

\textbf{Wichtige Technologien:}
\begin{itemize}[noitemsep]
    \item Agent-Frameworks: LangChain, LangGraph
    \item Generative Modelle: Gemini API
\end{itemize}

\subsection*{(Optional) 5. Qualitätssicherungs- und Ethik-Agent}

\textbf{Hauptaufgaben:}
\begin{itemize}[noitemsep]
    \item Prüfung der faktischen Korrektheit und Quellenintegrität.
    \item Sicherstellung ethischer Richtlinien und Qualitätsstandards.
\end{itemize}

\textbf{Wichtige Technologien:}
\begin{itemize}[noitemsep]
    \item Moderations-API: GPT Moderation API
    \item Evaluations-Tools: Hugging Face Evaluation Suite, BERT-Klassifikatoren
\end{itemize}

\section*{Ablaufplan für Studierende (Agile Methodik)}

\subsection*{Woche 1:}
\begin{itemize}[noitemsep]
    \item Datenerhebung und Vorbereitung der IR-Dokumente (2020–2024).
    \item Erste Implementierung der multimodalen Einbettungen und des RAG-Systems.
    \item Implementierung der RAG- und Analyse-Agenten.
\end{itemize}

\subsection*{Woche 2:}
\begin{itemize}[noitemsep]
    \item Integration des Web-Agenten (Echtzeitdaten).
    \item Implementierung des Koordinations-Agenten.
    \item (Optionales) Fine-tuning der Modelle.
    \item UI-Entwicklung mit Gradio; abschließende Qualitätssicherung und Deployment.
\end{itemize}

\section*{Abschließende Abgaben}
\begin{itemize}[noitemsep]
    \item Fertige App (Gradio UI, Hugging Face Spaces)
    \item Dokumentiertes GitHub-Repository
    \item Agile Projektdokumentation mit Jira
    \item Abschlusspräsentation mit Live-Demo
    \item Technischer Bericht (Systemarchitektur, Entscheidungen, Erfahrungen)
\end{itemize}

\section*{Lernziele und berufliche Relevanz}

Studierende erwerben praxisorientierte Fähigkeiten, die direkt auf berufliche Rollen im Bereich Finanzanalysen und Generative KI vorbereiten:
\begin{itemize}[noitemsep]
    \item Multimodale Informationsbeschaffung und wissensbasierte Antwortgenerierung
    \item Fortgeschrittene Finanzanalysen und Prognosemodelle
    \item Web Scraping und API-Integration für Echtzeitdaten
    \item Agile Teamarbeit und Projektmanagement
    \item Praxisorientierte Implementierung und UI-Entwicklung
\end{itemize}

Dieses Projekt spiegelt realitätsnah Systeme wider, die aktuell in der Industrie eingesetzt werden.

\section*{Weiterführende Ressourcen}

\begin{itemize}[noitemsep]
    \item \href{https://langchain-ai.github.io/langgraph/tutorials/workflows/}{Workflows and Agents}
    \item \href{https://langchain-ai.github.io/langgraph/tutorials/multi_agent/agent_supervisor/}{Multi-agent Supervisor}
    \item \href{https://blog.langchain.dev/semi-structured-multi-modal-rag/}{Multi-Vector Retriever for Multimodal RAG}
    \item \href{https://python.langchain.com/docs/how_to/multimodal_inputs/}{Passing Multimodal Data to Models}
    \item \href{https://langchain-ai.github.io/langgraph/concepts/multi_agent/}{Multi-agent Systems Concepts}
    \item \href{https://python.langchain.com/docs/how_to/qa_sources/}{Returning Sources in RAG}
    \item \href{https://python.langchain.com/docs/how_to/qa_citations/}{Adding Citations in RAG Applications}
\end{itemize}

\end{document}
